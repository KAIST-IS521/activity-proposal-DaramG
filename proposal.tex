\documentclass[a4paper, 11pt]{article}

\usepackage{kotex} % Comment this out if you are not using Hangul
\usepackage{fullpage}
\usepackage{hyperref}
\usepackage{amsthm}
\usepackage[numbers,sort&compress]{natbib}

\theoremstyle{definition}
\newtheorem{exercise}{Exercise}

\begin{document}
%%% Header starts
\noindent{\large\textbf{IS-521 Activity Proposal}\hfill
                \textbf{HyungSeok Han}} \\
         {\phantom{} \hfill \textbf{DaramG}} \\
         {\phantom{} \hfill Due Date: April 15, 2017} \\
%%% Header ends

\section{Activity Overview}

제대로 된 ELF parser를 만드는 것이 목표이다.

objdump, readelf 를 비롯한 대부분의 ELF parser 들은 corrupted 된 ELF header 에 대해서는 parsing을 하지 못하는 현상을 보인다.

심지어 최근 readelf 에서는 Code execution vulnerability 까지 존재했었다.

따라서 corrupted 된 ELF header에 대해서도 parsing을 잘 수행하는 ELF parser를 만드는 것이 목표이다.

\section{Exercises}

\begin{exercise} Parsing and print format


  ELF는 크게 ELF header, Program header, Section header의 header를 갖는다.
  각각 header를 parsing 하여, 각 헤더의 종류, 존재하는 offset, 정보들을 일정 형식에 맞추어서 출력해주는
  ELF parser를 만든다.

\end{exercise}

\begin{exercise} Parsing target

  Compiler 를 통해서 만든 정상적인 binary에 대해서는 당연히 parsing이 진행되어야 하고,
  corrupted 된 ELF 들을 예시로 주어 이것들 또한 parsing 할 수 있도록 한다.

\end{exercise}

\begin{exercise}  Recover headers

  2번에서 주어진 corrupted 된 ELF 들을 readelf, objdump 등과 같은 툴로도 parsing이 되는
  ELF를 만들 수 있도록 한다.

\end{exercise}

\section{Expected Solutions}

주어진 test case들을 모두 통과하는 ELF parser를 만들길 기대한다.
또한 corrupted 된 ELF를 복구할 수 있길 기대한다.

\bibliography{references}
\bibliographystyle{plainnat}

\end{document}
