\documentclass[a4paper, 11pt]{article}

\usepackage{kotex} % Comment this out if you are not using Hangul
\usepackage{fullpage}
\usepackage{hyperref}
\usepackage{amsthm}
\usepackage[numbers,sort&compress]{natbib}

\theoremstyle{definition}
\newtheorem{exercise}{Exercise}

\begin{document}
%%% Header starts
\noindent{\large\textbf{IS-521 Activity Proposal}\hfill
                \textbf{HyungSeok Han}} \\
         {\phantom{} \hfill \textbf{DaramG}} \\
         {\phantom{} \hfill Due Date: April 15, 2017} \\
%%% Header ends

\section{Activity Overview}

제대로 된 ELF parser를 만드는 것이 목표이다.

objdump, readelf 를 비롯한 대부분의 ELF parser 들은 corrupted 된 ELF header 에 대해서는 parsing을 하지 못하는 현상을 보인다.

따라서 corrupted 된 ELF header에 대해서도 parsing을 잘 수행하는 ELF parser를 만드는 것이 목표이다.

\section{Exercises}

Describe a series of exercises that students will carry out. (학생들이 하게
될 연습문제를 순차적으로 서술.)

\begin{exercise}

  In this exercise, you do ...

\end{exercise}

\begin{exercise}

  In this exercise, you do ...

\end{exercise}

\begin{exercise}

  In this exercise, you do ...

\end{exercise}

\section{Expected Solutions}

주어진 test case들을 모두 통과하는 ELF parser를 만들길 기대한다.


\bibliography{references}
\bibliographystyle{plainnat}

\end{document}
